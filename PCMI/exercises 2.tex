\documentclass[12pt]{report}

\usepackage{enumitem}
%\setlist{noitemsep}
\usepackage{amsfonts}
  \usepackage{amsmath,amsthm}
\usepackage[margin=1.5cm,bottom=2cm]{geometry}

\newbox\sqrtvelubox
\setbox\sqrtvelubox\hbox{$\sqrt{\vphantom{0}}$\'elu}
\newcommand{\sqrtvelu}{\copy\sqrtvelubox}


\usepackage{hyperref}
\hypersetup{
    colorlinks=true,
    linkcolor=blue,
    filecolor=magenta,      
    urlcolor=magenta,
    pdftitle={Overleaf Example},
    pdfpagemode=FullScreen,
    }




\newcommand{\F}{\mathbb{F}}
\newcommand{\Fpbar}{\overline{\mathbb{F}}_p}
\newcommand{\Fqbar}{\overline{\mathbb{F}}_q}
\newcommand{\Gal}{\mathrm{Gal}}
\newcommand{\N}{\mathbb{N}}
\newcommand{\OO}{\mathcal{O}}
\newcommand{\EE}{\mathcal{E}}
\newcommand\OOu[1]{\OO_{\!{#1}}}
\newcommand{\QQ}{\mathcal{Q}}
\newcommand{\RR}{\mathcal{R}}
\newcommand{\II}{\mathcal{I}}
\newcommand{\JJ}{\mathcal{J}}
\newcommand{\mA}{\mathcal{A}}
\newcommand{\Q}{\mathbb{Q}}
\newcommand{\R}{\mathbb{R}}
\newcommand{\Z}{\mathbb{Z}}
\newcommand{\C}{\mathbb{C}}
\newcommand{\Ell}{\mathcal{E}\hspace{-0.065cm}\ell\hspace{-0.035cm}\ell}
%\renewcommand{\i}{\mathbf{i}}
%\renewcommand{\j}{\mathbf{j}}
\renewcommand{\ij}{\mathbf{ij}}
\newcommand{\ji}{\mathbf{ji}}

\newcommand\Zsqrt[1]{\Z[\hspace{-.1em}\sqrt{#1}]}
\newcommand\Qsqrt[1]{\Q(\!\sqrt{#1})}
\newcommand\Leg[2]{\left( \frac{#1}{#2}\right)}
\newcommand{\cl}{\operatorname{cl}}
\newcommand{\Aut}{\operatorname{Aut}}
\newcommand{\End}{\mathrm{End}}
\newcommand{\Endp}{\mathrm{End}_p}
\newcommand{\id}{\mathrm{id}}

\newcommand{\afrak}{\mathfrak{a}}
\newcommand{\bfrak}{\mathfrak{b}}
\newcommand{\cfrak}{\mathfrak{c}}
\newcommand{\gfrak}{\mathfrak{g}}
\newcommand{\kfrak}{\mathfrak{k}}
\newcommand{\lfrak}{\mathfrak{l}}
\newcommand{\mfrak}{\mathfrak{m}}
\newcommand{\pfrak}{\mathfrak{p}}
\newcommand{\rfrak}{\mathfrak{r}}
\newcommand{\tfrak}{\mathfrak{t}}
\newcommand{\ufrak}{\mathfrak{u}}
\newcommand{\xfrak}{\mathfrak{x}}
\newcommand{\yfrak}{\mathfrak{y}}
\newcommand{\zfrak}{\mathfrak{z}}
\newcommand{\sfrak}{\mathfrak{s}}
\newcommand{\Pfrak}{\mathfrak{P}}
\newcommand{\tm}{T_m}
\newcommand{\Pinf}{\mathbf{0}}

\newcommand\norm[1]{\mathrm{N}(#1)}
\newcommand\trace{\operatorname{tr}}

\newcommand\im{\operatorname{im}}
\newcommand\Frob{\operatorname{Frob}}

\newcommand{\hooklongrightarrow}{\lhook\joinrel\longrightarrow}
\newcommand{\hooklongleftarrow}{\longleftarrow\joinrel\rhook}
\DeclareRobustCommand\twoheadlongrightarrow{\relbar\joinrel\twoheadrightarrow}

\newcommand{\infpt}{\infty} % point at infinity
\DeclareMathOperator{\poly}{poly}
\DeclareMathOperator{\ord}{ord}
\DeclareMathOperator{\val}{val}
\DeclareMathOperator{\cond}{cond}



  \newtheorem{theorem}{Theorem}[section]
  \newtheorem{lemma}[theorem]{Lemma}
  \newtheorem{conj}[theorem]{Conjecture}
  \newtheorem{cor}[theorem]{Corollary}
  \newtheorem{prop}[theorem]{Proposition}
  \newtheorem{claim}[theorem]{Claim}
  \newtheorem{definition}[theorem]{Definition}
  \newtheorem{fact}[theorem]{Fact}
  \newtheorem{remark}[theorem]{Remark}
  \newtheorem{problem}[theorem]{Problem}
  \newtheorem{question}[theorem]{Question}
  \newtheorem{example}[theorem]{Example}
  \newtheorem{warning}[theorem]{Warning}
  
  \newcommand{\Cl}{\operatorname{Cl}}
  \newcommand{\Hom}{\operatorname{Hom}}
  \newcommand{\GL}{\operatorname{GL}}
  \newcommand{\tr}{\operatorname{tr}}

  \newcommand{\nrd}{\operatorname{nrd}}
  \newcommand{\FpGraph}{{\mathcal{G}_\ell(\mathbb{F}_p)}}
  \newcommand{\FpSubGraph}{\mathcal{S}}
  \newcommand{\FpBarGraph}{{\mathcal{G}_\ell(\overline{\mathbb{F}_p})}}
  
\newcommand{\FpBarGraphtwo}{{\mathcal{G}_2(\overline{\mathbb{F}_p})}}
\newcommand{\FpGraphtwo}{{\mathcal{G}_2(\mathbb{F}_p})}
\newcommand{\FpGraphthree}{{\mathcal{G}_3(\mathbb{F}_p})}
\newcommand{\FpBarGraphthree}{{\mathcal{G}_3(\overline{\mathbb{F}_p})}}
\newcommand{\jp}{{\mathbf{j}}}
\newcommand{\jpp}{{\mathbcal{j}}}

\newcommand{\trd}{\operatorname{trd}}
\newcommand{\FF}{\mathbb{F}}

\newcommand{\CC}{\mathbb C}
\newcommand{\Fp}{\mathbb{F} _p}
\newcommand{\Glk}{\mathcal{G}_{\ell}(K)}
\newcommand{\Glfpbar}{\mathcal{G}_{\ell}(\overline{\mathbb{F}}_p)}
\newcommand{\Glfp}{\mathcal{G}_{\ell}(\mathbb F_p)}

\newcommand{\Res}{\operatorname{Res}}
\newcommand{\Tr}{\operatorname{Tr}}


\renewcommand{\O}{\mathcal{O}}
\newcommand{\disc}{\operatorname{disc}}
\newcommand{\different}{\mathcal{D}(\O) }

\begin{document}

\thispagestyle{empty}

\hyphenpenalty 8000
\sloppy
\centerline{\Large\bf PCMI 2021: Supersingular isogeny graphs in cryptography } 

\vspace{4mm}
\centerline{\Large\bf Exercises Lecture 2: Quaternion algebras, Endomorphism rings} 
\vspace{4mm}
\centerline{\bf TA: Jana Sot\'{a}kov\'{a}  \\  version \today }

\smallskip

\begin{enumerate}
\item (Quaternion algebras and orders) For small primes $p $, define the quaternion algebra $B := B_{p, \infty} = \Q\langle 1, i, j, k\rangle $ with $i^2 = -r $ and $j^2 = -p$ and $ij=-ji =k$: 
\begin{enumerate}
\item Use \verb=QuaternionAlgebra< RationalField() | -r, -p >=;
\item For $p \equiv 3 \bmod 4$, use $-r = -1$;
\item For $p \equiv 5 \bmod 8$, use $-r = -2$;
\item Otherwise, find $r$ as a prime $r\equiv 3 \bmod 4$ such that $\left( \frac{r}{p} \right) = -1$. 
\end{enumerate}

Verify that $B$ is only ramified at $p$ and infinity. Verify that $i^2 =-r$ and $j^2 = -p$. Find the norm, trace and the minimal polynomial of the element $w = 2+i-3j + 4k$.


\item (Maximal orders) Write down a maximal order in each of the quaternion algebras.
\begin{enumerate}
\item Using the Magma command \verb=MaximalOrder=; 
\item Using a basis and \verb=QuaternionOrder=;
\end{enumerate}

Find the discriminant and the norm form of the maximal order.

\item For $p= 67$, take any maximal order $\O \subset B_{p, \infty}$. Then:
\begin{enumerate}
\item Enumerate all the left-ideal classes in $\O$;
\item For every ideal class, pick a representative and find the right order of the ideal;
\item Compute the norm of all these ideals;
\item Figure out which of these maximal orders correspond to elliptic curves defined over $\mathbb{F}_p$. Show that the following suffices: \label{curves_fp}
\begin{enumerate}
\item Compute the norm form of these maximal orders;
\item Find out whether they represent $p$;
\end{enumerate}
Check the count by looking at how many supersingular $j$-invariants there are in $\mathbb{F}_p$.


 
\end{enumerate}

\item (Matching endomorphism rings to supersingular elliptic curves) For $p = 67$, determine the endomorphism rings of all supersingular elliptic curves defined over $\mathbb{F}_{p^2}$:
\begin{enumerate}
\item List all the maximal orders in $B_{p, \infty}$;
\item List all the supersingular $j$-invariants;
\item  Start from an elliptic curve with `known' endomorphism ring, e.g. $E : y^2 = x^3 - x$;
\item For small $\ell$, compare the $\ell$-isogenies between the elliptic curves and ideals of norm $\ell$. 
Use \eqref{curves_fp} to narrow down the orders for elliptic curves defined over $\F_p$.
\end{enumerate}

% $ (1+pi)/2 = ker 2 \subset ker (1+pi); ker((1+pi)/2) = [2]ker(1+pi) $

%Embed the quaternion algebra into the matrix ring M_2(Q(sqrt(a)))
\item (Quaternion algebras and Matrix rings) To add

\end{enumerate}

\end{document}
