\documentclass[12pt]{report}

\usepackage{enumitem}
%\setlist{noitemsep}
\usepackage{amsfonts}
  \usepackage{amsmath,amsthm}
\usepackage[margin=1.5cm,bottom=2cm]{geometry}

\newbox\sqrtvelubox
\setbox\sqrtvelubox\hbox{$\sqrt{\vphantom{0}}$\'elu}
\newcommand{\sqrtvelu}{\copy\sqrtvelubox}


\usepackage{hyperref}
\hypersetup{
    colorlinks=true,
    linkcolor=blue,
    filecolor=magenta,      
    urlcolor=magenta,
    pdftitle={Overleaf Example},
    pdfpagemode=FullScreen,
    }




\newcommand{\F}{\mathbb{F}}
\newcommand{\Fpbar}{\overline{\mathbb{F}}_p}
\newcommand{\Fqbar}{\overline{\mathbb{F}}_q}
\newcommand{\Gal}{\mathrm{Gal}}
\newcommand{\N}{\mathbb{N}}
\newcommand{\OO}{\mathcal{O}}
\newcommand{\EE}{\mathcal{E}}
\newcommand\OOu[1]{\OO_{\!{#1}}}
\newcommand{\QQ}{\mathcal{Q}}
\newcommand{\RR}{\mathcal{R}}
\newcommand{\II}{\mathcal{I}}
\newcommand{\JJ}{\mathcal{J}}
\newcommand{\mA}{\mathcal{A}}
\newcommand{\Q}{\mathbb{Q}}
\newcommand{\R}{\mathbb{R}}
\newcommand{\Z}{\mathbb{Z}}
\newcommand{\C}{\mathbb{C}}
\newcommand{\Ell}{\mathcal{E}\hspace{-0.065cm}\ell\hspace{-0.035cm}\ell}
%\renewcommand{\i}{\mathbf{i}}
%\renewcommand{\j}{\mathbf{j}}
\renewcommand{\ij}{\mathbf{ij}}
\newcommand{\ji}{\mathbf{ji}}

\newcommand\Zsqrt[1]{\Z[\hspace{-.1em}\sqrt{#1}]}
\newcommand\Qsqrt[1]{\Q(\!\sqrt{#1})}
\newcommand\Leg[2]{\left( \frac{#1}{#2}\right)}
\newcommand{\cl}{\operatorname{cl}}
\newcommand{\Aut}{\operatorname{Aut}}
\newcommand{\End}{\mathrm{End}}
\newcommand{\Endp}{\mathrm{End}_p}
\newcommand{\id}{\mathrm{id}}

\newcommand{\afrak}{\mathfrak{a}}
\newcommand{\bfrak}{\mathfrak{b}}
\newcommand{\cfrak}{\mathfrak{c}}
\newcommand{\gfrak}{\mathfrak{g}}
\newcommand{\kfrak}{\mathfrak{k}}
\newcommand{\lfrak}{\mathfrak{l}}
\newcommand{\mfrak}{\mathfrak{m}}
\newcommand{\pfrak}{\mathfrak{p}}
\newcommand{\rfrak}{\mathfrak{r}}
\newcommand{\tfrak}{\mathfrak{t}}
\newcommand{\ufrak}{\mathfrak{u}}
\newcommand{\xfrak}{\mathfrak{x}}
\newcommand{\yfrak}{\mathfrak{y}}
\newcommand{\zfrak}{\mathfrak{z}}
\newcommand{\sfrak}{\mathfrak{s}}
\newcommand{\Pfrak}{\mathfrak{P}}
\newcommand{\tm}{T_m}
\newcommand{\Pinf}{\mathbf{0}}

\newcommand\norm[1]{\mathrm{N}(#1)}
\newcommand\trace{\operatorname{tr}}

\newcommand\im{\operatorname{im}}
\newcommand\Frob{\operatorname{Frob}}

\newcommand{\hooklongrightarrow}{\lhook\joinrel\longrightarrow}
\newcommand{\hooklongleftarrow}{\longleftarrow\joinrel\rhook}
\DeclareRobustCommand\twoheadlongrightarrow{\relbar\joinrel\twoheadrightarrow}

\newcommand{\infpt}{\infty} % point at infinity
\DeclareMathOperator{\poly}{poly}
\DeclareMathOperator{\ord}{ord}
\DeclareMathOperator{\val}{val}
\DeclareMathOperator{\cond}{cond}



  \newtheorem{theorem}{Theorem}[section]
  \newtheorem{lemma}[theorem]{Lemma}
  \newtheorem{conj}[theorem]{Conjecture}
  \newtheorem{cor}[theorem]{Corollary}
  \newtheorem{prop}[theorem]{Proposition}
  \newtheorem{claim}[theorem]{Claim}
  \newtheorem{definition}[theorem]{Definition}
  \newtheorem{fact}[theorem]{Fact}
  \newtheorem{remark}[theorem]{Remark}
  \newtheorem{problem}[theorem]{Problem}
  \newtheorem{question}[theorem]{Question}
  \newtheorem{example}[theorem]{Example}
  \newtheorem{warning}[theorem]{Warning}
  
  \newcommand{\Cl}{\operatorname{Cl}}
  \newcommand{\Hom}{\operatorname{Hom}}
  \newcommand{\GL}{\operatorname{GL}}
  \newcommand{\tr}{\operatorname{tr}}

  \newcommand{\nrd}{\operatorname{nrd}}
  \newcommand{\FpGraph}{{\mathcal{G}_\ell(\mathbb{F}_p)}}
  \newcommand{\FpSubGraph}{\mathcal{S}}
  \newcommand{\FpBarGraph}{{\mathcal{G}_\ell(\overline{\mathbb{F}_p})}}
  
\newcommand{\FpBarGraphtwo}{{\mathcal{G}_2(\overline{\mathbb{F}_p})}}
\newcommand{\FpGraphtwo}{{\mathcal{G}_2(\mathbb{F}_p})}
\newcommand{\FpGraphthree}{{\mathcal{G}_3(\mathbb{F}_p})}
\newcommand{\FpBarGraphthree}{{\mathcal{G}_3(\overline{\mathbb{F}_p})}}
\newcommand{\jp}{{\mathbf{j}}}
\newcommand{\jpp}{{\mathbcal{j}}}

\newcommand{\trd}{\operatorname{trd}}
\newcommand{\FF}{\mathbb{F}}

\newcommand{\CC}{\mathbb C}
\newcommand{\Fp}{\mathbb{F} _p}
\newcommand{\Glk}{\mathcal{G}_{\ell}(K)}
\newcommand{\Glfpbar}{\mathcal{G}_{\ell}(\overline{\mathbb{F}}_p)}
\newcommand{\Glfp}{\mathcal{G}_{\ell}(\mathbb F_p)}

\newcommand{\Res}{\operatorname{Res}}
\newcommand{\Tr}{\operatorname{Tr}}


\renewcommand{\O}{\mathcal{O}}
\newcommand{\disc}{\operatorname{disc}}
\newcommand{\different}{\mathcal{D}(\O) }

\begin{document}

\hyphenpenalty 8000
\sloppy

\centerline{\Large\bf David Kohel's lectures  \\  Quaternions (\today)}

\smallskip


\begin{description}
\item[The bilinear map and quadratic form]
Symmetric bilinear map on $\O$ is given as  \[\langle x, y \rangle = \norm{x+y}-\norm{x}-\norm{y}\] 
define the associated Gram matrix
\[ (a_{ij} = (\langle x_i, x_j \rangle ))\]
for a given basis $x_1, \dots, x_4$. This has discriminant $N^2$ for $N$ being the discriminant of the quaternion algebra.

\item[Lemma 1.4] If $x, y$ are elements such that $\Q[x] \neq \Q[y]$ and $D_1, D_2$ are the discriminants of $\Z[x], \Z[y]$, then for $T = \Tr(x)\Tr(y) - 2 \Tr(xy)$ we have
\[ [x,y] = xy - yx = - \frac{D_1D_2 -T^2 }{4}\]

We want $4N = D_1 D_2 - T^2 $. The discriminants come from determinants of certain matrices with the inner products. 


\item[Successive minima] $a_k = $smallest value $a$ such that $\{ x \in \O : \langle x,x \rangle \leq a \}$ generates a submodule of rank $k$. By constructions, this is attained at some elements. 

There are various bounds on the successive minima, in terms of the Hermite constant $\gamma_n$. For us, we just need $\gamma_4^4=  4$ (and later $\gamma_2, \gamma_3$ in proofs) and so \[ \det(\O) = N^2 \leq \prod a_k \leq \gamma_n^n \det(\O) = 4  N^2 .\] 

From this and the values being successive minima, we can write down regions in which those values need to live. Throw more math at it to get sharper regions.

 Convenient to consider $r_i = \log_{N/4}(\norm{x_i})$ for $x_i$ the element attaining the minimum $a_i$.
 
 \item[Discriminant module] is the module $\O / \Z$ (considered as a $\Z$-module). This can be identified with $\Z \wedge \O \subset \Lambda^2 (\O)$. The determinant on this module is given as 
 \[ \det(x \wedge y) = \det \left( \begin{matrix}
 \langle x, x \rangle & \langle y,x \rangle \\ \langle x, y \rangle & \langle y, y \rangle
\end{matrix}  \right) \]
whose restriction to $1 \wedge x $ satisfies $\det(1 \wedge x ) = |\disc(\Z[x])|$. And so $\det(m \wedge x) = \det(1 \wedge x)$ for $m \in \Z$. 

The discriminant module has determinant $4N^2$. If $\O = \Z \langle 1, x, y, z\rangle$ then the discriminant is spanned by $1 \wedge x, 1 \wedge y, 1 \wedge z$. So if we want to figure out the quadratic form of this quadratic module, we just need to figure out the matrix of the bilinear map:

 \[ \det( 1 \wedge y_1, 1 \wedge y_2) = \det \left( \begin{matrix}
 \langle 1, 1 \rangle & \langle y_1, 1 \rangle \\ \langle 1, y_2 \rangle & \langle y_1, y_2 \rangle
\end{matrix}  \right) \]

This allows us - in principle, or so far with a lot of pain - to write down all the Gram matrix of the discriminant module.

Moreover, we can figure out the regions in which the successive minima need to lie. Writing $D_1, D_2, D_3$ for the successive minima and $s_i$ for their $\log_{4N}(D_i)$, then we can write down domains in which they lie. 

\item[Prop 1.9] The band $1 \leq s_1 + s_2 \leq 1< 1 + \log_{4N}{4/3}$ contains only points asociated to maximal orders with an element of norm $N$. 

In this region, we get $D_1 D_2 - T^2 = 4N$ and from Lemma 1.4 we get an actual element of norm $N$ as the commutator.


\item[Dual discrimimant module] the different is \[ \mathcal{D}(\O) = \{ x \in B : \Tr(x\O) \subset \Z \}^{-1}\]
and we define the DDM as 
\[ \frac{1}{N}  \left( \Z \wedge \Z [\mathcal{D}(\O) ] \right).\]

The different has the property that every suborder of $\Z[\different]$ has discriminant divisible by $N$ so: 
\begin{center}
This module represent $m$ $\leftrightarrow$ there is a subring of $\O$ of discriminant $-mN$ 
\end{center}
Advantages: DDM has discriminant $4N$, so smaller than the discriminant module?

Again can write down regions in which the (first two) successive minima need to lie. 

\item[Lemma 1.13] The logarithmic points associated to the DDM  lie on the lines \[  t_i = \log_{4N}(m)\] for $-mN$ a discriminant. (this seems like $i = 1$).

\item[Maximal orders with an embedding of order of discriminant $-p$] are exactly those whose DDM represents $1$. 

\item[Maximal order with an embedding of order of disc $-4p$] represent $4$, but their first successive min can be $1$ or $3$, which can happen (example in the notes).


\item[Section 2] Shows that from the data $D_1, D_2, T$ we can reconstruct the maximal order. Assuming that the orders with discriminant $D_1, D_2$ are optimally embedded in $\O$, generated by $\alpha, \beta$, then 

\begin{itemize}
\item The quaternion order $\Z \langle \alpha, \beta \rangle$ has discriminant $D_1D_2 - T^2$;
\item the commutator $[\alpha, \beta]$ gives a quadratic subring of discriminant $D_1D_2 - T^2 $.
\end{itemize}


\item[Norm forms, ideals, $\ell$-adic representation]

Normalized ideal norm $q_I : I \rightarrow \Z$ is independent of the ideal class representative and has discriminant $N^2$.

\item[Lemma 3.2] Determine representations of powers of $\ell$ by $q_I$ by using the knowledge of the $\ell$-adic zeros:

The surface $S = V(q_I) \subset \mathbb{P}(I\otimes \Z_\ell) = \mathbb{P}^3$ is isomorphic to $\mathbb{P}^1 \times \mathbb{P}^1$ over $\Z_\ell$.

\begin{proof}
Locally, all ideals are principal, and the reduced norm is independent of the class set representative, so replace $I_\ell$ with $M_2 (\Q_\ell)$ and the norm form with the determinant. So then zeros of the form are the determinant zero matrices, ie $\{(a:b:c:d) : ad-bc = 0\}$, which is the image of the Serge embedding. 
\end{proof} 

\item[How to lift the solution to $I$] 

\end{description}


\end{document}
