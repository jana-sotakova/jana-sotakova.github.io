\documentclass[12pt]{report}

\usepackage{enumitem}
%\setlist{noitemsep}
\usepackage{amsfonts}
  \usepackage{amsmath,amsthm}
\usepackage[margin=1.5cm,bottom=2cm]{geometry}

\newbox\sqrtvelubox
\setbox\sqrtvelubox\hbox{$\sqrt{\vphantom{0}}$\'elu}
\newcommand{\sqrtvelu}{\copy\sqrtvelubox}


\usepackage{hyperref}
\hypersetup{
    colorlinks=true,
    linkcolor=blue,
    filecolor=magenta,      
    urlcolor=magenta,
    pdftitle={Overleaf Example},
    pdfpagemode=FullScreen,
    }




\newcommand{\F}{\mathbb{F}}
\newcommand{\Fpbar}{\overline{\mathbb{F}}_p}
\newcommand{\Fqbar}{\overline{\mathbb{F}}_q}
\newcommand{\Gal}{\mathrm{Gal}}
\newcommand{\N}{\mathbb{N}}
\newcommand{\OO}{\mathcal{O}}
\newcommand{\EE}{\mathcal{E}}
\newcommand\OOu[1]{\OO_{\!{#1}}}
\newcommand{\QQ}{\mathcal{Q}}
\newcommand{\RR}{\mathcal{R}}
\newcommand{\II}{\mathcal{I}}
\newcommand{\JJ}{\mathcal{J}}
\newcommand{\mA}{\mathcal{A}}
\newcommand{\Q}{\mathbb{Q}}
\newcommand{\R}{\mathbb{R}}
\newcommand{\Z}{\mathbb{Z}}
\newcommand{\C}{\mathbb{C}}
\newcommand{\Ell}{\mathcal{E}\hspace{-0.065cm}\ell\hspace{-0.035cm}\ell}
%\renewcommand{\i}{\mathbf{i}}
%\renewcommand{\j}{\mathbf{j}}
\renewcommand{\ij}{\mathbf{ij}}
\newcommand{\ji}{\mathbf{ji}}

\newcommand\Zsqrt[1]{\Z[\hspace{-.1em}\sqrt{#1}]}
\newcommand\Qsqrt[1]{\Q(\!\sqrt{#1})}
\newcommand\Leg[2]{\left( \frac{#1}{#2}\right)}
\newcommand{\cl}{\operatorname{cl}}
\newcommand{\Aut}{\operatorname{Aut}}
\newcommand{\End}{\mathrm{End}}
\newcommand{\Endp}{\mathrm{End}_p}
\newcommand{\id}{\mathrm{id}}

\newcommand{\afrak}{\mathfrak{a}}
\newcommand{\bfrak}{\mathfrak{b}}
\newcommand{\cfrak}{\mathfrak{c}}
\newcommand{\gfrak}{\mathfrak{g}}
\newcommand{\kfrak}{\mathfrak{k}}
\newcommand{\lfrak}{\mathfrak{l}}
\newcommand{\mfrak}{\mathfrak{m}}
\newcommand{\pfrak}{\mathfrak{p}}
\newcommand{\rfrak}{\mathfrak{r}}
\newcommand{\tfrak}{\mathfrak{t}}
\newcommand{\ufrak}{\mathfrak{u}}
\newcommand{\xfrak}{\mathfrak{x}}
\newcommand{\yfrak}{\mathfrak{y}}
\newcommand{\zfrak}{\mathfrak{z}}
\newcommand{\sfrak}{\mathfrak{s}}
\newcommand{\Pfrak}{\mathfrak{P}}
\newcommand{\tm}{T_m}
\newcommand{\Pinf}{\mathbf{0}}

\newcommand\norm[1]{\mathrm{N}(#1)}
\newcommand\trace{\operatorname{tr}}

\newcommand\im{\operatorname{im}}
\newcommand\Frob{\operatorname{Frob}}

\newcommand{\hooklongrightarrow}{\lhook\joinrel\longrightarrow}
\newcommand{\hooklongleftarrow}{\longleftarrow\joinrel\rhook}
\DeclareRobustCommand\twoheadlongrightarrow{\relbar\joinrel\twoheadrightarrow}

\newcommand{\infpt}{\infty} % point at infinity
\DeclareMathOperator{\poly}{poly}
\DeclareMathOperator{\ord}{ord}
\DeclareMathOperator{\val}{val}
\DeclareMathOperator{\cond}{cond}



  \newtheorem{theorem}{Theorem}[section]
  \newtheorem{lemma}[theorem]{Lemma}
  \newtheorem{conj}[theorem]{Conjecture}
  \newtheorem{cor}[theorem]{Corollary}
  \newtheorem{prop}[theorem]{Proposition}
  \newtheorem{claim}[theorem]{Claim}
  \newtheorem{definition}[theorem]{Definition}
  \newtheorem{fact}[theorem]{Fact}
  \newtheorem{remark}[theorem]{Remark}
  \newtheorem{problem}[theorem]{Problem}
  \newtheorem{question}[theorem]{Question}
  \newtheorem{example}[theorem]{Example}
  \newtheorem{warning}[theorem]{Warning}
  
  \newcommand{\Cl}{\operatorname{Cl}}
  \newcommand{\Hom}{\operatorname{Hom}}
  \newcommand{\GL}{\operatorname{GL}}
  \newcommand{\tr}{\operatorname{tr}}

  \newcommand{\nrd}{\operatorname{nrd}}
  \newcommand{\FpGraph}{{\mathcal{G}_\ell(\mathbb{F}_p)}}
  \newcommand{\FpSubGraph}{\mathcal{S}}
  \newcommand{\FpBarGraph}{{\mathcal{G}_\ell(\overline{\mathbb{F}_p})}}
  
\newcommand{\FpBarGraphtwo}{{\mathcal{G}_2(\overline{\mathbb{F}_p})}}
\newcommand{\FpGraphtwo}{{\mathcal{G}_2(\mathbb{F}_p})}
\newcommand{\FpGraphthree}{{\mathcal{G}_3(\mathbb{F}_p})}
\newcommand{\FpBarGraphthree}{{\mathcal{G}_3(\overline{\mathbb{F}_p})}}
\newcommand{\jp}{{\mathbf{j}}}
\newcommand{\jpp}{{\mathbcal{j}}}

\newcommand{\trd}{\operatorname{trd}}
\newcommand{\FF}{\mathbb{F}}

\newcommand{\CC}{\mathbb C}
\newcommand{\Fp}{\mathbb{F} _p}
\newcommand{\Glk}{\mathcal{G}_{\ell}(K)}
\newcommand{\Glfpbar}{\mathcal{G}_{\ell}(\overline{\mathbb{F}}_p)}
\newcommand{\Glfp}{\mathcal{G}_{\ell}(\mathbb F_p)}

\newcommand{\Res}{\operatorname{Res}}
\newcommand{\Tr}{\operatorname{Tr}}


\renewcommand{\O}{\mathcal{O}}
\newcommand{\disc}{\operatorname{disc}}
\newcommand{\different}{\mathcal{D}(\O) }

\begin{document}

\thispagestyle{empty}

\hyphenpenalty 8000
\sloppy
\centerline{\Large\bf PCMI 2021: Supersingular isogeny graphs in cryptography } 

\vspace{4mm}
\centerline{\Large\bf Exercises Lecture 1: Elliptic curves, Isogenies, CGL Hash Function} 
\vspace{4mm}
\centerline{\bf TA: Jana Sot\'{a}kov\'{a}  \\  version \today }

\smallskip

Use Magma to do the following exercises. If you need help to get started, please ask on the Discord!
\begin{enumerate}
\item (Elliptic curves) Over $\mathbb{F}_p$ for $p = 431$:
\begin{enumerate}
\item Define an elliptic curve $E/ \mathbb{F}_p$ with $E: y^2 = x^3 +x$.
\item Compute its $j$-invariant;
\item Find an elliptic curve $E_1 / \mathbb{F}_p$ with $j$-invariant $234$;
\item Is this elliptic curve supersingular?
\item Find another elliptic curve $E_2$ with $j$-invariant $234$. Are $E_1$ and $E_2$ isomorphic over $\F_p$? Can you find a non-isomorphic such pair? Hint\footnote{Quadratic twists.}
\end{enumerate}
\item (Isogenies) Compute the following for $E: y^2 = x^3 +x / \mathbb{F}_{431^2}$
\begin{enumerate}
\item Isogeny $ \varphi: E  \rightarrow E'$ with kernel generated by $(0,0)$. What is the degree?
\item Compute the dual isogeny $\hat{\varphi} : E' \rightarrow E$;
\item Find all the isogenies of degree $2$ from $E$.
\item Find all the cyclic isogenies of degree $16$ from $E$.
\item Compute a cyclic isogeny of degree $16$ as a sequence of $2$-isogenies.
\end{enumerate}
\item (Modular polynomial) Use the modular polynomial $\Phi_N(X,Y)$ to find isogenous curves:
\begin{enumerate}
\item Find all the $2$-isogenies curves to $E : y^2 = x^3 + 26x + 279/ \mathbb{F}_{431^2}$;
\item Find $j$-invariants of elliptic curves admitting a $16$-isogeny from $E$. Hint\footnote{To deal with the large coefficients, reduce the polynomial to $\mathbb{F}_{p^2}$}
\item Find all the self-loops in the $\ell$-isogeny graph for $\ell \leq 11$. 
\end{enumerate}

\item (Supersingular isogeny graphs) Write code to generate the supersingular isogeny graph over $\mathbb{F}_{p^2}$, using the following steps. On input coprime  primes $p$  and $\ell$;
\begin{enumerate}
\item Find one supersingular elliptic curve over $E_0/ \mathbb{F}_{p^2}$, represented by the $j$-invariant;
\item Write a neighbor function that on input an elliptic curve $E$, finds all the neighbours of $E$ in the SSIG $\mathcal{G}_\ell$: (the $j$-invariants) all the supersingular elliptic curves $\ell$-isogenous to $E$.
\item Using a breadth-first-search approach, generate the graph by starting from the curve found in Step (b) and the Neighbor function from Step (c).
\end{enumerate}
\item (If you've done Exercise 4), for primes $p \equiv 1 \bmod 12$, find the adjacency matrix $A$ of the SSIG and find the diameter. SSIGs have very short diameters.
\item (\href{https://eprint.iacr.org/2006/021.pdf}{CGL Hash function}) For a small prime $p$ and any starting supersingular elliptic curve $E$, find a collision for the CGL hash fuction on the $2$-isogeny SSIG. I.e., find two strings that hash to the same elliptic curve. Hint\footnote{Requires you to decide on the ordering of the edges in the SSIG. Find two isogenies to the same curve.}
\end{enumerate}



\end{document}
